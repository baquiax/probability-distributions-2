\chapter{Distribución Hipergeométrica}

Este modelo presenta similitudes con el Binomial, \textit{pero sin la suposición de independencia de éste último}. \cite{wiki:2} \cite{hyp:1}

\section{Descripción}

Esta distribución está relacionada con muestreos aleatorios y sin reemplazom], a diferencia de la \textit{Binomial}. La idea general es que de la muestra total con \textit{N} elementos, existe un subconjunto con \textit{d} elementos que pertenecen a una categoría cualquiera \textit{A}, siendo el resto perteneciente a otra categoría \textit{B}.

\section{Variable aleatoria}
La variable aleatoria en cuestión, 	al igual que en la binonail, se define como la cantidad de \textit{éxitos} que se den.

La función de probabilidad para la distribución hipergeométrica, puede deducirse a través del análisis combinatorio.

\section{PDF}
\begin{center}
	$P(X=\textit{x})	= {\frac{ {d \choose x} {N - d \choose n-x} }{ {N \choose n} }} $
\end{center}

\subsection{Parámetros}
Los parámetros que usamos en las funciones de esta distribución son:

\begin{center}
	\begin{tabular} {| l | l |}
		\hline
		N & Cantidad total de elementos de la población. \textit{N} $\in 0, 1, 2, 3, ...$\\ \hline
		d & elementos con éxito en la población. \textit{m} $\in 0, 1, 2, 3, ...$ \\ \hline
		n & Tamaño de la muestra extraída. \textit{n} $\in 0, 1, 2, 3, ...$ \\ \hline
		x & la cantidad de éxitos de los n extraídos. \\ \hline
	\end{tabular}
\end{center}

El elemento ${d \choose x}$, evidencia la relación binomial.

\section{CDF}
La función de distribución acumualada es:

\begin{center}
	$P(X <= \textit{x})	= 1 - {\frac{ {d \choose x + 1} {N - d \choose n- x -1} }{ {N \choose n} }} $
\end{center}

\section{Media y Varianza}
\subsection{Media}
La esperanza de una variable aleatoria X con distribución hipergeométrica es:
\begin{center}
	$E(X) = {\frac{n d}{N}}$
\end{center}

\subsection{Varianza}
siendo su varianza
\begin{center}
	$Var(X) = (\frac{N - n}{N - 1})(\frac{nd}{N})(1 - \frac{d}{N})$
\end{center}

\section{MGF}

\section{Gráficas}

\section{Aplicaciones en la vida real}
